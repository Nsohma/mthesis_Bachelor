\documentclass[11pt,oneside,openany,report]{jsbook}

\usepackage[a4paper,truedimen,margin=25truemm]{geometry}
\usepackage{cscover}
\usepackage[dvipdfmx]{graphicx}
\usepackage[nobreak]{cite}
\usepackage[a4paper,dvipdfmx,pdfdisplaydoctitle=true,%
    bookmarks=true,bookmarksnumbered=true,bookmarkstype=toc,bookmarksopen=true,%
    pdftitle={学位論文の体裁に関する研究},%
    pdfauthor={科学大太郎}%
    ]{hyperref}
\usepackage{pxjahyper}

\renewcommand{\bibname}{参考文献}
\setcounter{tocdepth}{2}
\pagestyle{plain}

\newcommand{\TODO}[1]{\textbf{[TODO: #1]}}
%\renewcommand{\TODO}[1]{}

\thesistype{学士特定課題研究論文}
\title{日本のアニメーション制作における\\リテイクの調査及び蓄積と活用}
\author{新穂 壮真}
\studentid{22B30610}
\affiliation{東京科学大学\\情報理工学院\\情報工学系} 
\date{2026年1月}

\supervisorname{指導教員}
\supervisor{齋藤 豪}
%\dsupervisorname{副指導教員}
%\dsupervisor{工学 次郎}

\begin{document}

\frontmatter
\maketitle

\chapter{概要}
ここに概要を書きます。

\tableofcontents
\listoffigures
\listoftables

%%%%%%%%%%%%%%%%%%%%%%%%%%%%%%%%%%%%%%%%%%%%%%%%%%

\mainmatter
\chapter{序論}
\section{本論文の背景と目的}

近年、アニメーション制作のデジタル化が進み、作画や撮影、編集
といった各工程においてコンピュータ上での作業が一般的になってきている。
それに伴い制作過程で扱うデジタルデータの量も増加している。
%これに伴い、制作過程で生成されるレイアウト、原画、動画、仕上げ画像など、
%さまざまなデータをどのように蓄積し利活用するかが重要な課題となっている。
とくに日本のアニメーション制作では、セルアニメーション由来の分業制やカット単位の工程管理が
現在も広く採用されており、各工程で多数の中間生成物や指示書類が発生し、
作品制作の過程で複雑に行き交っている。
その一方で、これらのデータが制作後に体系的に整理・蓄積されることは少なく、
制作ノウハウや現場での判断の多くが個人の経験や属人的な管理に依存しているのが現状である。

こうした状況を受けて、これまでにアニメーション制作におけるデータ管理や
中間生成物の蓄積を目的とした研究・システム開発がいくつか行われてきた。
しかし、それらはいずれも修正という行為そのもの、
すなわち演出や作画監督による修正指示や、その結果として更新された原画・動画といった情報を、
データ活用の観点から蓄積することには十分な焦点を当てていない。
新潟大学アニメ中間素材データベース AIMDB\cite{aimdb} はアニメ制作過程におけるデータの蓄積を目的としたデータベースであり、
リテイクをはじめとした様々な素材の蓄積に対応しているが、データ活用を意識した蓄積構造が設計されているとは言い難い。
渕上ら\cite{fucci}の研究では、データ活用を見据えた中間生成物の蓄積を行っているが、
修正に関する蓄積には対応していない。
修正は、作品のクオリティや作業者ごとの傾向を端的に反映する情報でありながら、
多くの場合は各カットの中に埋もれた形で扱われ、体系的に蓄積・分析されてこなかったと言える。

そこで本論文では、実際のアニメーション制作現場で役立つ視点から、
修正の蓄積とその活用方法を提案することを目的とする。
%その蓄積と利活用を主な対象とする。具体的には、原画や動画に対する演出修正・
%作画監督修正などの作画上の修正情報を、従来のカットや中間生成物の管理とは
%独立した形で収集・構造化し、それらを分析することで制作進行を円滑にするための
%システムを提案することを目的とする。
この目的を達成するため、アニメーション制作現場における修正の運用実態を調査し、
どのような単位・粒度で修正を蓄積すべきかを整理する。その上で
修正とともに蓄積すべきタグについて設計する。
次に、その結果に基づき、修正の内容や発生箇所、担当者、工程などの情報を体系的に
蓄積するためのデータ構造を設計し、修正データを分析することで
制作進行の計画立案や作業者配置の検討を支援する分析システムを実装する。
%最後に、提案システムを用いて修正データの傾向分析や可視化を行い、
%その結果から制作進行の計画立案や人員配置の最適化といった場面における有用性を検証する。





\section{本論文の構成}
本論文は6章から構成されている。各章の内容は以下の通りである。

\vspace{\baselineskip}

\noindent
\textbf{1章 序論}

本章では、本研究の背景と目的について述べる。

\vspace{\baselineskip}

\noindent
\textbf{2章 関連研究}

本章では、関連研究としてアニメーション制作に使用されるツールやアニメーション制作の
現状、及び画像認識技術やLLMについて述べる。

\vspace{\baselineskip}

\noindent
\textbf{3章 アニメ会社への調査}

本章では、アニメーション会社に対して行った調査の方法と結果について述べる。

\vspace{\baselineskip}

\noindent
\textbf{4章 提案手法}

本章では、アニメーション会社への調査結果を踏まえて、
本研究で提案するシステムの設計と実装について述べる。

\vspace{\baselineskip}

\noindent
\textbf{5章 評価}

本章では、本研究で提案するシステムの実際の使用事例とフィードバック、
アニメーション会社の方からの意見によりシステムの有用性について述べる。


\vspace{\baselineskip}

\noindent
\textbf{6章 結論}

本章では、本論文のまとめ及び今後の課題について述べる。


%%%%%%%%%%%%%%%%%%%%%%%%%%%%%%%%%%%%%%%%%%%%%%%%%%

\chapter{関連研究}
\section{はじめに}
本章では、本研究に関連する従来研究として、日本のアニメーション制作においてデータ管理を目的とした
ツールや中間生成物の蓄積に関する研究について述べる。最後に、本研究を実現するための
既存技術である画像認識技術やLLMについて述べる。

\section{アニメーション制作のデータ蓄積と活用方法の現状}
本節では、日本のアニメーション制作のデータ蓄積の現状と検討されている活用方法について述べる。

\subsection{アニメーションアーカイブの現状2017\cite{yamakawa}}
山川氏は、株式会社プロダクション・アイジーにおいてアーカイブグループのリーダを務め、
業務中に発生したアニメーション制作資料の収集・整理・選別・保管と、その利活用への対応を主な業務としている。
記事では、2017 年時点のアニメーションアーカイブの現状として、
社内スタッフからの問い合わせが月最大 200 件にも達し、
その内容が新作や続編制作のための設定画・色指定データの取り出し、
アニメータや演出家に仕事を発注する際の作風確認、新人教育用の原画貸出、商品化や映像配信、展示企画のための資料提供など多岐にわたることが紹介されている。
このような事例から、過去の原画や設定資料、中間成果物といった制作資料が、
制作・教育・商品開発などさまざまな用途において日常的に参照されており、
その活用意義が非常に大きいことがわかる。

一方で、原画類や中間成果物、契約書類、映像マスター、デジタルデータなど、
多様な媒体・形式の資料が大量に発生するため、その全体像や関係性を把握できる人材は限られており、
評価選別や保管には大きなコストがかかることも指摘されている。
その結果、多くの制作会社では原画を映像納品後に産業廃棄物として処分してしまうなど、
資料の大部分が十分な整理を経ずに失われている現状がある。
山川氏は、制作現場で長年培われてきた番号付けや書類管理のルールを基盤としつつ、
アーカイブズ学の知見を取り入れて「アーカイブや利活用時に困らない管理方法」を整備し、
会社や業界を超えた連携によってアニメーション資料の収集・選別・提供を継続的に行える枠組み
を構築することの重要性を述べている。この議論は、アニメーション制作過程で生じる中間生成物
や修正に関する情報についても、将来の制作や教育に資する形で蓄積・整理しておくための視点
が求められていることを示唆している。


\subsection{アイデアソンによるアニメーション中間生成物の活用可能性の検討\cite{matsushita}}
松下らは、日本のアニメーション制作がごく限られた人で行われているために余裕がないことから、
情報の整理が十分に行われていないとしている。
しかし、これらの整理されていない中間生成物がアニメーションの文化資本であることに注目しており、
”コストをかけてでもこれらの中間生成物をアーカイブする”
というモチベーションをアニメータに与えるために、2022年に中間生成物の利用用途
と活用可能性をアイデアソンにより検討している。

このアイデアソンでは、国立情報学研究所が公開しているリトルウィッチアカデミアの素材 
(トリガーデータセット)\cite{trigger} を活用しており、計 27 名の学生が参加した。
全 7 チームでの議論を複数日にまたがって行い、チームごとの成果をそれぞれ報告するという流れで行われた。

報告での活用事例の案として、モブキャラ作成システム、新人の自主練習支援システム、
原画データベースシステムなど複数の項目が挙げられた。
これらの挙げられた案は大きく 3 つの観点 (中間生成物の横断的検索の必要性、
機械学習リソースとしての活用、新たな表現メディア創出の手がかり) にまとめられており、
アイデアソンではこれら 3 つの観点をデータ活用の結論としている。
しかし、今回のアイデアソンではユーザ視点での活用について不十分だとしており、
データの関係性が整理されて保存されていれば更なる活用案が創出されると
松下らは期待をしているようだ。
関係性が整理されたデータは現在存在しないため、
そういった関係性を整理したデータの蓄積が求められている。


\section{アニメーション制作におけるデータ管理用ツール}
この節では、日本のアニメーション制作で用いられているデータ管理用ツールについて
述べる。

\subsection{flow production tracking(旧shot grid)\cite{shotgrid}}
flow production tracking は AUTODESK 株式会社が作成したツールであり、
映像・CG・アニメーション制作、ゲーム開発のプロジェクト管理を目的に作成されている。
アニメーション制作の中間生成物の管理もプロジェクト管理と合わせてできることから、
中間生成物のオンライン上の保存が可能である。

メイン機能であるプロジェクト管理は、立場に応じたタスクのスケジュールの管理をする。
図 2.3 のように、ガントチャートによりスケジュールを表示するため、
アニメーション制作の管理に適している。アップロードした中間生成物の扱いやすいような
フィルタ付けや、セルの背景色の変更をすることもでき、
中間生成物をシステム内で扱う自由度は高い。コミュニケーションのためのメール機能も存在し、
中間生成物と合わせて様々な指示を送ることで、日々のタスクが容易に把握できる。

また、素材のレビュー機能として、図 2.4 のような各メディアに容易にメモを
書き込む機能が備えられており、視覚的な修正指示を与えることができる。
アニメーションや VFX、ゲーム開発のためのデジタルレビューツール RV \cite{rv} と組み合わせて使うことで、
複数フォーマットや多様な解像度で映像を 再生することも可能であり、
映像比較やコメントなど多様な機能を通して、アニメーション制作の充実したレビュー機能を担う。

さらに、図 2.5 のようにアセットの追跡機能もあり、
情報を追跡しやすくするために、自身のフィールドやステータスを追加することができる。
ユーザを選んでアクセス権を制御することもでき、誤ったデータ変更への対策がなされている。

上記の機能を有する flow production tracking は 900 以上のスタジオでの実績があり、1日に数千
ユーザ、数百万タスクへの対応が実証されており、その普及率は業界内でも非常に高い。 
しかし、多機能であるがゆえに初心者には扱いにくいという点もあり、直感的でない複雑なツールになってしまっている。
導入コストが高いことや、日本のアニメーション制作に使用されるタイムシートなどへの
対応ができないという問題点も残るのが現状だ。また、リテイクなどの処理に対応できず、
リテイクのための別名のカットを新しく作り直す必要があるため、本来の意図と異なる使い方を強いられる場合もある。
日本のアニメーション制作のいくつかの要件に適用していないように、保存された中間
生成物も日本のアニメーション制作に適した構造とはなっておらず、蓄積という観点で不十分である。

\subsection{save point\cite{savepoint}}
save point は株式会社 MUGENUP のプロジェクト管理ツールであり、
イラストや 3DCG、映像、広告アセット、アニメーションなど様々な
クリエイティブ制作に対応している。図 \ref{savePointSchedule} のように制作スケジュールや進捗
をツール内で管理することができ、情報共有を簡単に行える自由度の高いツールとなっている。

\begin{figure}[h]
    \centering
    \includegraphics[width=\linewidth]{fig/savePointSchedule.png}
    \caption{savepoint のスケジュール管理機能}\label{savePointSchedule}
\end{figure}

\begin{figure}[h]
    \centering
    \includegraphics[width=\linewidth]{fig/savePointThread.png}
    \caption{savepoint のスレッド機能}\label{savePointThread}
\end{figure}

また、図 \ref{savePointThread} のようなスレッド機能やプレビュー機能はプロジェクト管理の中で中間生成物を扱うこと
ができるため、中間生成物の確認や指示をする際に活用される。

ツール紹介ではコストの save、データの save、ディレクターを save の 3 つの save を挙げており、
制作進行担当者やディレクターの仕事の減少やデータの一元管理による人為的なミスの防止を掲げている。

しかし、save point はプロジェクト管理に重点を置いており、
データの活用を見据えているものではないため、中間生成物の蓄積には至っていない。
そのため、save point 内の機能にある検索も、欲しい中間生成物が明確である前提での検索となってしまい、
より自由度の高い検索法に応用するのが難しい。


\subsection{Hiero\cite{hiero}}
Hiero は Foundry 社が作成したツールであり、マルチショット管理や、編集、レビューワークフロー
に適している。マルチショットの管理は、制作の開始時点から納品まで制御し、中間生成物を扱える。
Hiero の特に優れている点はレビュー機能であり、プロジェクトのどの段階であっても中間生成物に
対して迅速なレビューが行える設計になっている。リアルタイム再生などの様々な機能の中でも、
オンライン編集の機能はアニメーション制作で使用されるエフェクトや背景をタイムライン上で扱うことが可能だ。
他の製品 (NUKE ファミリー製品) と組み合わせた利用方法もあり、共同作業の容易さがメリットとして挙げられる。
しかし、自由度の高いレビューという目的が強く、中間生成物を十分に整理できていない。そのため、
データ活用を見込むことは困難であり、蓄積手法としては不十分である。

\subsection{アニクロ\cite{anikuro}}
アニクロはメモリーテック株式会社のアニメーション制作管理ツールで、クラウド環境での中間生成物
の一元管理や、制作の進捗確認を行える。対応する制作工程は、絵コンテ、レイアウト、背景、原画、
動画、仕上げ、撮影、編集であり、日本のアニメーション制作の幅広い工程に対応している。各工程に対して、
中間生成物を保存できる仕組みになっており、各中間生成物のプレビューを行える。日本のアニメーション制作
で重視されるタイムシートについても、デジタルタイムシートの xdts 形式 \cite{xdts} を扱うことができ、
オンライン上での管理に対応している。また、修正指示等も中間生成物に手書きで書き込めるため、
コメントを含めた管理が可能だ。各工程の進捗は、カット表や日報表の形で表示され、日本のアニメーション制作
で使用されるリテイク指示にも対応している。

しかし、制作管理に重点をおいているため、中間生成物の構造的な蓄積はなされていない。また利用
するためには、各社それぞれの工程に対応するための相談を行う必要があり、普及率を急激に上げることは困難である。

\subsection{OLM FM tool\cite{olmfmtool}}
OLM FM tool は株式会社 OLM が作成するファイル管理を目的とするツールである。
アニメーショ ン制作の中間生成物を管理し、アニメータに共有することができる。
図 \ref{OLMFMTool} のような機能があり、ファイルのダウンロードアップロード、フォルダパスの作成、コピーアンドペースト、
削除など、自由度の高いディレクトリ操作が可能である。

\begin{figure}[h]
    \centering
    \includegraphics[width=\linewidth]{fig/OLMFMTool.png}
    \caption{OLM FM toolの自由度の高いディレクトリ操作機能}\label{OLMFMTool}
\end{figure}


GoogleDrive のセキュリティ面やアクセス制限を利用し、作品や話数、カット、工程などによって中間生成物
を管理している。データ紛失を起こすことなく、低速なネット環境でも快適に扱うことができる点が特徴的であり、
リモートワークへの対応力がある。

しかし、これらはファイル管理を行うのみで、データベースなどに中間生成物が蓄積されているわけではなく、
オンライン上に保存するだけになっている。また、ディレクトリをアニメーション作品ごとに作成する必要があり、
制作管理において時間の短縮への貢献があまりないという課題もある。

\subsection{Redmine\cite{redmine}}
Redmine はプロジェクト運営を支援するオープンソースソフトウェアであり、課題管理や情報共有に適している。
システムはWEB アプリケーションとして動作するため、複数人でどこからでも利用が可能である。
図 2.9 のチケット機能は課題を管理する機能であり、担当者や期日を含めて仕事の管理をする。
また、ガントチャート機能により、作業の進捗を視覚的に把握できるため、進捗管理を行いやすい。
版管理の機能もあり、git などのシステムと連携して、中間生成物の管理をすることもできる。

しかし、システム上に保存された中間生成物は構造的に蓄積されておらず、
中間生成物や版管理の情報を直接活かすことが難しい。

\subsection{株式会社サンジゲンの制作管理ツール\cite{sannjigenn}}
株式会社サンジゲン は 3DCG を用いたアニメーション制作を行うアニメーション会社の 1 つで
ある。株式会社サンジゲンの制作管理ツールは社内向けに社内エンジニアが作成しているもので、煩雑
なアニメーションの管理業務を自動化したいという想いから、このツールの作成を開始している。
制作の進行状況やステータスはもちろん、チャット機能やアップロード通知の機能など、
管理業務における多くの機能を有する。また、設定資料の共有もタグを用いて分散して管理されており、
探しやすい工夫が施されている。レビューも中間生成物を見ながら行うことができ、
日本のアニメーション制作の工程に幅広く対応している。

しかし、管理業務の自動化を主な目的としているため、
データの活用に関しては不透明な状況であるそうだ。中間生成物の保存はできているものの、
構造的な蓄積はできていない。


\section{アニメーションの中間生成物の蓄積に関する研究}
この節では、日本のアニメーション制作における中間生成物の蓄積に関する研究について述べる。

\subsection{新潟大学アニメ中間素材データベースAIMDB\cite{aimdb}}
新潟大学アニメ中間素材データベース AIMDB は、絵コンテ/原画/動画などを含む、
中間素材のアーカイブができるデータベースである。このデータベースの研究目的は、
中間生成物のアーカイブ化と データ処理、分析、セル画の保存ソリューション開発であり、
蓄積に焦点を当てている。実際に蓄積したいくつかの中間生成物の公開がなされている。
アーカイブされた中間生成物はそれぞれの属性をもとに検索ができ、
以下の図 2.10 のような形で作品の題名や会社、中間生成物の種類によって絞り込むことができる。

しかし、このデータベースには入力インタフェースが提案されておらず、
アニメータが中間生成物を蓄積することはできない。そのため、中間生成物を蓄積するためには、
過去の制作で利用した中間生成物をスキャンしてデータ化した後に技術者が
データベースに蓄積する必要がある。また、タイムシートの取り扱いが紙のみであるため、
デジタル形式 (XDTS \cite{xdts}) に対応していない。
検索に関しても求めている中間生成物の詳細が判明していないと困難であり、
自由度の高い検索を行うためには十分な蓄積手法であるとは言えない。


\subsection{実務化に聞くアニメアーカイブデータベースの可能性と課題
新潟大学アニメ中間素材データベース(AIMDB)へのフィードバックから\cite{matsumoto}}
新潟大学アニメアーカイブ研究センターでは過去のアニメーション作品の中間生成物のアーカイブを
進めており、先述した AIMDB \cite{aimdb} として閲覧者を限定して公開している。
この研究では、アニメアーカイブを促進する山川道子氏とシニアプロデューサーを務める
野口光一氏が AIMDB を使用した所感をまとめている。



山川氏と野口氏から挙げられた意見は、ニーズに応じて解像度を調整してはどうかという意見や、
検索だけでなく資料を読み解くリファレンス能力が必要などの意見があった。
また、現場に共有されずに削除される資料の存在を補う工夫(絵コンテからのキーワード抽出など)
の必要性や、適切にデジタル化できるアーキビストの必要性など課題も挙げられた。

これらの意見を基にすると、アニメータが注目する点や説明文などのアニメータの行動を含めて、
中間生成物を蓄積することを考慮する必要がある。また、中間生成物を適切に蓄積できるアーキビストが
いなくとも、アニメータ自身が中間生成物を蓄積できる設計が求められている。


\subsection{長尾らのデータベース\cite{nagao}}
セルアニメーション由来の手描きを含む制作手法を用いる日本のアニメーション制作では分業制が用いられており、
1 カットに対して携わる人数が多い。また、制作現場ごとに制作プロセスが異なるため、
業界内での画一的な中間生成物の管理は困難であり、スタジオごとに各々の手法で中間生成物を管理している。
しかし、画一的にデータを蓄積できなければ、中間生成物の活用を見込むことは難しく、貴重とされる中間生成物
が無駄となる。

これらの事実を受けて、長尾らはアニメーション制作事後の中間生成物を調査し、
画一的な管理手法としてリレーショナルデータベースを提案した。具体的には、2014 年に株式会社スタジオ
コロリドと株式会社ロボットが共同制作を行った”マルコメ株式会社の「料亭の味」の CM アニメーション” 
\cite{marukome} のうち、「単身赴任編」と「夜食編」の 2 つのエピソードを対象に調査を行っている。
中間生成物を含む プロジェクトディレクトリの調査の結果、以下の 5 点の課題を挙げている。

\begin{enumerate}
    \item ディレクトリの構成や命名規則がはっきりしておらず、必要となるファイルの在処を把握するのが困難
    \item 同一内容のファイルが複数のディレクトリに点在
    \item 工程の進捗状況を知るためには、複雑なデータ構造の確認が必要
    \item チェックの意図が不明瞭
    \item データが残されていないファイルが存在
\end{enumerate}


これら 5 点に対して、長尾らは課題に適した設計指針を示している。設計指針としては、カット単位でデータ
を扱うこと、データの種類をタイムシートに関わるものとそうでない場合を分別できること、 
タイムシートの情報ではフレームとレイヤの情報を持てること、修正前後全てのデータが確認できること
が挙げられている。カット単位で中間生成物を管理することは、カット袋での管理が基本であるアニメーション
業界内では標準的であり、同一のファイルが点在するようなことを防ぐメリットがある。また、
工程ごとの管理を行うことによって、必要となるファイルの在処を明確にし、進捗状況をつかめるようになる。

これらの設計指針を基準に作成したデータベースが以下の図 2.11 の 22 のテーブルからなる
リレーショナル・データベースである。カット、ファイル、作業の 3 つのテーブル群をフレームやレイヤなど
のテーブルで結びつけ、カットやファイル、作業を関連付けている。また、登録日時と更新日時を蓄積することで、
作業の履歴を管理することができる構造となっている。

長尾らのデータベースは日本のアニメーション特有の制作手法に特化した構造となっており、
管理が難しい日本のアニメーション中間生成物を適切に扱えるデータベースである。


\subsection{夏らのシステム\cite{yiyi}}
夏らは先述した長尾らのデータベース \cite{nagao} に対して、
原画と動画の中間生成物を蓄積できる WEB システムを提案している。
このシステムはアニメータ自身がアニメーション制作中に中間生成物を蓄積できる設計になっているため、
制作後にアップロードするという手間を省くことができる。

中間生成物を蓄積する具体的な方法は、WEB インタフェースの作業ページ上でのアップロードである。
アップロードの形式はレイヤ別の絵とタイムシートを含めたディレクトリ構造を zip 形式に圧縮
したものであり、作画ツールとして主流である CLIP Studio Paint \cite{clipstudio} で
描いた絵やタイムシートを容易にその形式にできることから、連携も容易である。アップロード
により蓄積した中間生成物は図 2.12 に示す作業画面にて確認をすることができる。
作業ページのインタフェースはカット袋やタイムシートなど、
日本のアニメーション制作で使用される管理手法と似た表を意識して作られており、
アニメータが親近感を持って使うことができる設計になっている。
また、単に中間生成物をアップロードするだけでなく、下記のような制作支援を提供することで、ア
ニメータが制作途中に中間生成物を蓄積するモチベーションを高めている。

\begin{enumerate}
    \item タイムシートの編集 \\アップロードされたタイムシートはインタフェース上で編集を行うことができる。タイムシートのセル値の入力や削除、セル値の移動、undo、redo、コピー、ペーストなどができ、ユーザが容易に編集可能である。タイムシートの編集ができるツールは東映デジタルタイムシート \cite{toei} など限られていることから貴重である。
    \item 映像作成 \\インタフェース上のタイムシートとアップロードされた原画や動画を使用して、現状の映像を作成することができる仕組みとなっている。映像作成は線画に対応しており、レイヤの重ね合わせはもちろん、指定したレイヤだけで映像を作成することもできる。
    \item 版による履歴管理 \\アニメータは自身の作業の中で、絵を描き換えやタイムシートを編集などの変更を加える。夏らのシステムではこの変更の際に、素材の上書きをせずに、過去の版の内容を復元できる手法である。具体的には、システム上での映像作成時や再アップロード時に自動で過去の版を登録する。この蓄積方法により、過去の版の中間生成物を復元することが可能となり、インタフェース上で確認できる。版の保存は映像確認を行った際に、ユーザ自身の要望に応じて行うことができる。
    \item 版を用いた映像比較 \\過去の素材を版として蓄積しているため、過去の版時点での映像をインタフェース上で作成できる。また、過去の版を複数選択すれば、図 2.13 に示すインタフェースで、最大 4 つの版の映像を同時再生により比較できる。この映像比較機能によって、アニメータがより良い素材を選択することができることに加え、アニメータの試行錯誤の過程やその結果をデータベースに蓄積できる。試行錯誤のアーカイブは、正解例のみならず失敗例を蓄積できるため、試行錯誤意図の理解へ役立つ可能性がある。
\end{enumerate}

このような制作支援を通して、多忙なアニメーション業界での使用可能性を高めている。
蓄積するタイミングが制作途中であるために、最終盤の中間生成物のみでなく、途中経過の中間生成物を蓄積
できることも大きな強みである。また、このシステムは WEB システムであるので、アニメータが使用する環境
を構築するのが容易であり、導入難易度が比較的低いシステムでもある。

\subsection{渕上らのシステム\cite{fucci}}
\section{既存関連ツールの比較}
\subsection{比較指標}
\subsection{比較結果}
\section{画像認識技術の比較}
\section{おわりに}

結論は、網羅的にかつ簡潔に。

%%%%%%%%%%%%%%%%%%%%%%%%%%%%%%%%%%%%%%%%%%%%%%%%%%
\chapter{アニメ会社への調査}
\section{はじめに}
\section{調査目的/調査方法}
\section{アニメーション制作工程でのワークフロー}
\section{各工程で発生するリテイク}
\subsection{原画}
\subsection{動画}
\subsection{彩色}
\subsection{ラッシュ}
\section{リテイクに関する考察}
\subsection{リテイクの発生原因}
\subsection{リテイクの活用可能性}
\section{おわりに}

\chapter{提案手法}
\section{はじめに}
\section{リテイクを活かしたシステムの提案}
\subsection{リテイクのタグ分類}
\subsection{制作進行を円滑にするための
アニメータのタグ別リテイク率計算システム}
\subsection{修正指示の補助資料としての修正前後の素材検索システム}
\section{リテイクのデータベース設計}
\section{実装方針}
\section{その他実装}
\section{おわりに}


\chapter{評価}
\section{はじめに}
\section{タグ付け精度の評価と考察}
\subsection{実験内容}
\subsection{実験結果}
\section{データベースの評価}
\subsection{制作進行からの意見}
\subsection{アニメータからの意見}
\section{検索システムの評価}
\subsection{制作進行からの意見}
\subsection{アニメータからの意見}
\section{おわりに}

\chapter{結論}
\section{本論文のまとめ}
\section{今後の課題}


\appendix
\chapter{リテイクに関与しない部分のアニメ会社調査結果}
必要に応じて、付録を載せる。

%%%%%%%%%%%%%%%%%%%%%%%%%%%%%%%%%%%%%%%%%%%%%%%%%%





\backmatter
\chapter{謝辞}
本論文の執筆にあたり、議論して頂いた関係者に感謝する。

%%%%%%%%%%%%%%%%%%%%%%%%%%%%%%%%%%%%%%%%%%%%%%%%%%

\bibliographystyle{jplain}
\bibliography{references}
%\begin{thebibliography}{99}
%  \bibitem{tokodai-xyz2015} 科学大太郎. 良い論文の書き方. \textit{Journal of XYZ}, Vol.~3, No.~4, pp. 15--34, 2015.
%\end{thebibliography}

\end{document}
